\documentclass{beamer}
%\usepackage{media9}

%\usepackage{hyperref}

% Necessary definitions:
%\setbeamersize{sidebar width left=0.5cm}
%\usepackage[dutch]{babel}
\usepackage{tikz}
%\usepackage{cancel}
\usepackage[asdasd]{optional}
\usepackage{mathrsfs}
\usepackage{bbm}%\ind
\usepackage[normalem]{ulem}
\usepackage{fancyvrb}
\usepackage{color}
\usepackage[latin1]{inputenc}
\usepackage{minted}
 % % % % % % % % % % % % % % % % % % %
\usepackage{xparse}
\usepackage{lineno}
\ExplSyntaxOn
\box_new:N \l_fvrb_box
\tl_new:N \l_fvrb_tl

\RenewDocumentCommand \FancyVerbFormatLine { m }
 {
   \hbox_set:Nn \l_fvrb_box { #1 }
    \dim_compare:nNnTF { \box_wd:N \l_fvrb_box }>{ \linewidth }
      {%box to big 
       \tl_set:Nn \l_fvrb_tl { #1 }
       \fvrb_use_tl:N \l_fvrb_tl
      } 
      {%box fits
       \box_use:N \l_fvrb_box
      }
 }

\cs_new:Npn \fvrb_use_tl:N  #1
 {
  \group_begin:
   \null\hfill\vbox_set:Nn \l_fvrb_box
     {\hsize=\linewidth
      \renewcommand\thelinenumber
           {
             \ifnum\value{linenumber}=1\relax\else
                  %$\rightarrow$
             \fi
           }
      \begin{internallinenumbers}
        \advance\hsize by -2em
        \hspace*{-2em}\tl_use:N #1
      \end{internallinenumbers}
     }
   \box_use:N \l_fvrb_box
  \group_end:
}

\ExplSyntaxOff

% % % % % % % % % % % % % % % %

\def\dotuline{\bgroup
  \ifdim\ULdepth=\maxdimen  % Set depth based on font, if not set already
   \settodepth\ULdepth{(j}\advance\ULdepth.4pt\fi
  \markoverwith{\begingroup
  \advance\ULdepth0.08ex
  \lower\ULdepth\hbox{\kern.15em .\kern.1em}%
  \endgroup}\ULon}

\def\dashuline{\bgroup
  \ifdim\ULdepth=\maxdimen  % Set depth based on font, if not set already
   \settodepth\ULdepth{(j}\advance\ULdepth.4pt\fi
  \markoverwith{\kern.15em
  \vtop{\kern\ULdepth \hrule width .3em}%
  \kern.15em}\ULon}

\mode<presentation>
{\usetheme{Boadilla} % This theme will be adjusted into the TUDelft lay-out
\setbeamercovered{transparent}}
\setbeamersize{text margin left=1cm}
\setbeamersize{text margin right=1cm}

\beamertemplatenavigationsymbolsempty
%---------------------------------------------------------------------------------
%  Take attention for the parts you may change. See the comment lines with: %>>>
%---------------------------------------------------------------------------------
%mo
%\usepackage{graphicx}

\setlength{\parskip}{5pt plus 2pt minus 1pt}
\sloppy

\providecommand{\cs}{,\,}
\providecommand{\NN}{\mathbb{N}}
\providecommand{\ZZ}{\mathbb{Z}}
\providecommand{\RR}{\mathbb{R}}
\providecommand{\QQ}{\mathbb{Q}}
\providecommand{\PP}{\mathbb{P}}

\providecommand{\E}{\mathbb{E}}
\providecommand{\e}{\mathrm{e}}
\renewcommand{\P}{\mathbb{P}} %overwriting P = paragraph
\providecommand{\eins}{\mathbf{1}}
\providecommand{\ms}[1]{\mathfrak{#1}}
\providecommand{\abs}[1]{\left\lvert#1\right\rvert}
\providecommand{\sabs}[1]{\left\langle #1\right\rangle}
\providecommand{\norm}[1]{\lVert#1\rVert}
\providecommand{\ind}[1]{{\mathbbm{1}_{#1}}}
\providecommand{\inv}{{-1}}
\providecommand{\iid}{{\ensuremath{\text{\emph{i.i.d.}}}}}
\newcommand{\dd}{{\,\mathrm d}}
\providecommand{\Inv}[1]{{\left(#1\right)}^{-1}}
\providecommand{\forme}[1]{\opt{forme}{\small {#1}}}

%\renewcommand{\complement}[1]{{#1}^{\subset}}
\renewcommand{\complement}[1]{{#1}^{\mathsf C}}
\renewcommand{\tilde}{\widetilde}
\renewcommand{\theta}{\vartheta}
\renewcommand{\phi}{\varphi}

\newcommand{\pin}[1]{{#1}^\star}

\newcommand{\nought}{\circ}
\providecommand{\trace}{{\operatorname{tr}}}

\newcommand{\Ell}{{\mathcal L}}
\renewcommand{\thefootnote}{\fnsymbol{footnote}}
%structure
\renewcommand{\cite}[2][1]{\nocite{#2}}


\definecolor{colora}{rgb}{0,0.8,0.81}
\definecolor{colorb}{rgb}{0.8,0.2,0}% #CF7100
\definecolor{colorc}{rgb}{0.2,0.2,0.8}% #CF7100
\newcommand{\cemph}[1]{{{\color{colorb} #1}}}
\newcommand{\blue}[1]{{{\color{colorc} #1}}}






%gets rid of bottom navigation bars
\setbeamertemplate{footline}[page number]{}
\setbeamertemplate{footline}[frame number]

%gets rid of navigation symbols
\setbeamertemplate{navigation symbols}{}


 
%>>> You may change the text in this part {Between brackets}:
%>>> This is for the Title page:
\newcommand*\titel{}
%\newcommand*\subkop{Introduction and provisoric outline}
\newcommand*\naam{}
\newcommand*\afdeling{TU Delft}
%>>> This is for the frame-title on the "Table of Contents" page:
\newcommand*\titelTOC{Outline}
%>>> This is for the frame-title on the "Table of Contents" page when the next subsection will start:
\newcommand*\subsectie{Next subsection}


%%% Not change this part below %%%
%%% Title Page (belongs to the theme)%%%
% Necessary part for the theme:
\title[]{XXX} % This title also appears in the TUDelft bar on the next pages
\subtitle{}
\author[M. Schauer]{Moritz Schauer}
%\institute[TUD/EUR]{\afdeling}
\date[Delft '14]{BaNaNa seminar, TU Delft, February, 2014}
%\AtBeginSubsection[]
%{\begin{frame}<beamer>\frametitle{\textbf{\LARGE{\textrm{\subsectie}}}}
%    \tableofcontents[currentsection,currentsubsection]  % Generation of the Table of Contents
%\end{frame}}
%\tikzset{textlabel/.style={color=white}}
\beamertemplatetransparentcovereddynamicmedium


\setbeamertemplate{frametitle}
{
\textbf{\Large{\textrm{\insertframetitle}}}
\par
}

%==============================================================
%%% Not change this part below, except maybe the folder where you placed the "TUDelft bies"
\begin{document}

%--------------------------------------------------------------
\begin{frame}
\begin{center}
{\large {\Huge ``} \hspace{4.0cm} \begin{minipage}[c]{15.5mm}\includegraphics[width=15.5mm]{img/logo_hires.png}\end{minipage}\hspace{4.0cm} {\Huge''} \\[2mm]
 A Fast Dynamic Language for Technical Computing 

\vspace{2cm}
Created by: Jeff Bezanson, Stefan Karpinski, Viral B. Shah, Alan Edelman et. al.

}
\end{center}
\end{frame}

\begin{frame}

{\large
\begin{center} \begin{minipage}[c]{15.5mm}\includegraphics[width=15.5mm]{img/logo_hires.png}\end{minipage}\end{center}

A sane and friendly programming language which allows
\begin{itemize}
\item  to write clean high level code 
\item and do fast low level number crunching
\end{itemize}
in one language within one framework.
}


\end{frame}


\begin{frame}\frametitle{Some stated design principles}
\begin{itemize}
\item Open source with an MIT licensed core
\item Dynamically typed with fast user-defined types
\item Multiple dispatch combined with a parametric type system
\item JIT compiler - fast vectorized and fast iterative code
\item Metaprogramming
\item Single environment to do technical computing and surrounding general programming tasks
\end{itemize}

\end{frame}

%\section{Friendly and flexible}
\begin{frame}[fragile]\frametitle{Example: Running average}
 
\begin{minted}[linenos,fontsize=\small]{julia}
X = cumsum(randn(10^6)) #random walk

# running average of order three, interactive style
runavg3(X) = [ (X[i-1] + X[i] + X[i+1])/3 for i=2:length(X)-1 ]

function runavg(X, d) #first shot at a generalization
   n = length(X)
   Y = similar(X, n - d + 1)
   Y[1] = mean(X[1:d])
   for i in 2:(n-d+1)
       Y[i] = Y[i-1] + 1/d * (X[i-1+d] - X[i-1])
   end
   Y
end
\end{minted}
 
\end{frame}

\begin{frame}[fragile]
Julia unimposingly computes the result very fast:

 
\begin{minted}[fontsize=\small]{julia}
julia> @elapsed runavg3(X)
0.003496773

julia> @elapsed runavg(X,3)
0.004001902

julia> @elapsed runavg(X,30)
0.004068611

julia> @elapsed cumsum(X)
0.004236988
\end{minted}
 
(Most of this is allocating the new array.)

\end{frame}

\begin{frame}[fragile]\frametitle{Type system}
Two uses

\begin{itemize}
\item Dynamic: Using types to dispatch the right method at runtime\\
\begin{minted}[fontsize=\small]{julia}
expm(A::HermOrSym) = (F = eigfact(A); F.vectors*Diagonal(exp(F.values))*F.vectors')
\end{minted}
\item  (Quasi) static: Helping the just-in-time compiler\\
\begin{minted}[fontsize=\small]{julia}
julia> xs = 1:5
julia>[i^3 for i in xs]
5-element Array{Any,1}
julia>[i::Int^3 for i in xs]
5-element Array{Int64,1}
\end{minted}

\end{itemize}


\end{frame}
 
\begin{frame}[fragile]\frametitle{Method dispatch}
 
\begin{minted}[fontsize=\small]{c}
#define MIN(a,b) (((a)<(b))?(a):(b))
#define MAX(a,b) (((a)>(b))?(a):(b))
\end{minted}
 
You remember?
%\end{frame}
%\begin{frame}\frametitle{Method dispatch}

\begin{itemize}
\item[] {\bf Static languanges:} Multiple dispatch, can follow a elaborated pattern, at compile time (function overloading.)
\item[] {\bf Dynamic languanges:} Single dispatch or no dispatch, simple, at run time.
\item[] {\bf Julia:} Elaborated, multiple dispatch with promotion rules, but does not slow down JIT'ed code. 
\end{itemize}
\end{frame}

\begin{frame}[fragile]\frametitle{Function overloading in C++}
%http://pic.dhe.ibm.com/infocenter/ratdevz/v8r5/index.jsp?topic=%2Fcom.ibm.tpf.toolkit.compilers.doc%2Fref%2Flangref_os390%2Fcbclr21018.htm
 You would not want to do this on runtime...
\begin{minted}[fontsize=\small]{text}
Argument-matching conversions occur in the following order:
An exact match, in which the actual arguments exactly match the type and number of formal arguments of one declaration of the overloaded function. This includes a match with one or more trivial conversions.
A match with promotions in which a match is found when one or more of the actual arguments is promoted
A match with standard conversions in which a match is found when one or more of the actual arguments is converted by a standard conversion
A match with user-defined conversions in which a match is found when one or more of the actual arguments is converted by a user-defined conversion
A match with ellipses 
\end{minted}
 
(OS/390 V2R10 C/C++ Language Reference)
\end{frame}


\begin{frame}[fragile]\frametitle{Multiple dispatch in Julia}
{\small
\begin{minted}[]{julia}
julia> methods(max)
# 14 methods for generic function "max":
max(x::Float64,y::Float64) at math.jl:334
max(x::Float32,y::Float32) at math.jl:335
max(x::BigFloat,y::BigFloat) at mpfr.jl:509
max{T<:Real}(x::T<:Real,y::T<:Real) at promotion.jl:191
max(x::Real,y::Real) at promotion.jl:172
max{T1<:Real,T2<:Real}(x::T1<:Real,y::AbstractArray{T2<:Real,N}) at operators.jl:247
max{T1<:Real,T2<:Real}(x::AbstractArray{T1<:Real,N},y::T2<:Real) at operators.jl:249
max{T1<:Real,T2<:Real}(x::AbstractArray{T1<:Real,N},y::AbstractArray{T2<:Real,N}) at operators.jl:253
max(x,y) at operators.jl:35
max(a,b,c) at operators.jl:67
max(a,b,c,xs...) at operators.jl:68
\end{minted}
}
\end{frame}



\begin{frame}[fragile]\frametitle{Metaprogramming}
%Compare http://hg.savannah.gnu.org/hgweb/octave/file/7eeaecac9b5b/libinterp/corefcn/data.cc
{\small
\begin{minted}[linenos]{julia}
for (fname, felt) in ((:zeros,:zero),
                      (:ones,:one),
                      (:infs,:inf), 
                      (:nans,:nan))
@eval begin
($fname){T}(::Type{T}, dims...)   = fill!(Array(T, dims...), ($felt)(T))
($fname)(dims...) = fill!(Array(Float64, dims...), ($felt)(Float64))
($fname){T}(x::AbstractMatrix{T}) = ($fname)(T, size(x, 1), size(x, 2))
end
end
\end{minted}
}
{
\small
\begin{minted}{julia}
julia> ones(3)
5-element Array{Float64,1}:
 1.0
 1.0
 1.0
\end{minted}
}

\end{frame}

\begin{frame}\frametitle{Word of caution}
\begin{itemize}
\item Young language, initial commit 2009, open source 2012
\item ``Fast moving target``
\item Memory hungry: Compiled and specialized methods
\item Limited debugging support
\end{itemize}
\end{frame}

\end{document}
